\documentclass[12pt]{article} % Документ принадлежит классу article, а также будет печататься в 12 пунктов.
\renewcommand{\labelenumii}{\theenumii}
\renewcommand{\theenumii}{\theenumi.\arabic{enumii}.}
\renewcommand{\labelenumiii}{\theenumiii}
\renewcommand{\theenumiii}{\theenumi.\arabic{enumii}.\arabic{enumiii}.}
 \usepackage[russian]{babel} % Пакет поддержки русского языка
 \usepackage{scalerel,amssymb}
 \def\circmark{\mathbin{\scalerel*{\circ}{j}}}
 \title{Контур безопсности} % Заглавие документа
 \date{16/10/2019} % Дата создания

 \begin{document}
 \maketitle

   \begin{enumerate}

  \item Введение
    \begin{enumerate} 
        \item \begin{large} \textbf{Цели} \end{large} \newline
          Этот документ определяет спецификацию требований к программному обеспечению (SRS) для системы обеспечения контура безопасности. В нем описываются область действия системы, как функциональные, так и нефункциональные требования к программному обеспечению, конструктивные ограничения и системные интерфейсы.
          Продукт должен обеспечить корректное функционирование для любого коммерческого предприятия.

        \item Границы применения
        \item \begin{large} \textbf{Определения, сокращения, термины} \end{large} \newline
          \begin{tabular}{ | l | l |}
            \hline
            Аббривиатура & Расшифровка \\ \hline
            СОКБ & Система обеспечения контура безопасности \\
            АРМ & Автоматизированное рабочее место \\
            ЗПБ & Зона повышенной безопасности \\
            УД & Уровень доступа \\
            КСБ & Комплекс системы безопасности \\
            СУРВ & Система учета рабочего времени \\
            СКУД & Система контроля и управления доступом \\
            \hline
          \end{tabular}
        \item Ссылки
        \item Краткий обзор
    \end{enumerate} 
  \item  \begin{large} \textbf{Общее описание} \end{large} \newline
  	\begin{enumerate}
  		\item \begin{large} \textbf{Описание изделия} \end{large} \newline
        Система обеспечения контура безопасности -система, представляющая собой комплекс программных и технических средств, необходимых для поддержания санкционированного доступа в помещения в охраняемых зонах.
        СОКБ должна предоставить доступ лицу, имеющему соответствующий пропуск. Лица, не имеющие пропуска, не должны иметь возможности доступа к объекту.  
        Программное обеспечение должно функционировать на двух уровнях безопасности. Первый уровень безопасности представляет собой установленный при входе в помещение турникет, считывающий с пропуска информацию о прошедшем лице. Вторым уровнем безопасности являются доступ к рабочему отделу организации. Также как и на первом уровне считывается информация о лице. При наличии соответствующих прав должен предоставиться доступ к отделу. Информация о доступах на обоих уровнях безопасности должна заноситься в базы данных. 
        Использование данной системы должно значительно снизить риск несанкционированного доступа, способного нанести серьезный материальный ущерб. 

  			\begin{enumerate}
  				\item \begin{large} \textbf{Интерфейсы системы} \end{large} \newline
            В качестве основного интерфейса взаимодействия с системой используется набор сетевых запросов, позволяющий выполнять следующие команды: \newline
            $\circmark $ получать полную статистику для конкретного пользователя, обьекта \newline
            $\circmark $ изменять текущие права доступа для конкретного пользователя, обьекта \newline
            $\circmark $ производить регистрацию новых пользователей \newline
            $\circmark $ прочие, специальные команды \newline
            Обязательным параметром каждого запроса является специальный токкен, подтверждающий права пользователя на выполноние команды. Такие токены имеею небольшой период действия и должны постоянно перевыпускаться. Их перевыпуском занимается администратор системы и рассылает по некому защищенному каналу.
  				\item \begin{large} \textbf{Интерфейсы пользователя} \end{large} \newline
            Обычные пользователи не должны взаимодействовать с системой. Процесс получения карты доступа и подача уведомлений о ее потере или истечению срока действия происходит вне системы. \newline
            Супер пользователи взаимодействуют с системой через вэб портал, который является графической оберткой над интерфейсами системы. Дополнительные интерфейсы включают электронную почту, прямое обращение. 
          \item \begin{large} \textbf{Интерфейсы аппаратных средств ЭВМ} \end{large} \newline
          Используется набор сетевых запросов, генерирующие извещений о следующих событиях: \newline
            $\circmark $ пользователь произвел открытие входа из вне контура безопсности   \newline
            $\circmark $ произошло закрытие входа \newline
            $\circmark $ пользователь произвел открытие входа изнутри контура безопасности \newline
            $\circmark $ зафиксировано присутсвие внутри контура безопасности \newline
            $\circmark $ зафиксировано ЧП внутри контура безопасности \newline
  				\item Интерфейсы программного обеспечения
  				\item Интерфейсы коммуникаций
  				\item Ограничения памяти
  				\item Действия
  				\item Требования настройки рабочих мест
  			\end{enumerate}
  		\item Функции изделия
  		\item Характеристики пользователей
  		\item Ограничения
  		\item Предложения и зависимости
  		\item Поднаборы требований(распределение требований)
  	\end{enumerate}
  \item Детальные требования
  	\begin{enumerate}
  		\item Внешние интерфейсы
  		\item Функции
  		\item Требования исполнения
  		\item Требования логики базы данных
  		\item Ограничения проекта
  		\item Характеристики программного обеспечения системы
  			\begin{enumerate}
  				\item Надежность
  				\item Эксплуатационная готовность
  				\item Безопасность
  				\item Ремонтопригодность
  				\item Переносимость
  			\end{enumerate}
  		\item Структурирование детальных требований
  			\begin{enumerate}
  				\item Режим системы
  				\item Классы пользователей
  				\item Объекты
  				\item Особенности
  				\item Воздействие
  				\item Реакция
  				\item Функциональные иерархии
  				\item Дополнительные комментарии
  			\end{enumerate}
  	\end{enumerate}
\end{enumerate}
 \end{document}