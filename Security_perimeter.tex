\documentclass[12pt]{article} % Документ принадлежит классу article, а также будет печататься в 12 пунктов.
\renewcommand{\labelenumii}{\theenumii}
\renewcommand{\theenumii}{\theenumi.\arabic{enumii}.}
\renewcommand{\labelenumiii}{\theenumiii}
\renewcommand{\theenumiii}{\theenumi.\arabic{enumii}.\arabic{enumiii}.}
 \usepackage[russian]{babel} % Пакет поддержки русского языка
 \usepackage{scalerel,amssymb}
 \def\circmark{\mathbin{\scalerel*{\circ}{j}}}
 \title{Контур безопсности} % Заглавие документа
 \date{16/10/2019} % Дата создания

 \begin{document}
 \maketitle

   \begin{enumerate}

  \item Введение
    \begin{enumerate} 
        \item \begin{large} \textbf{Цели} \end{large} \newline
          Этот документ определяет спецификацию требований к программному обеспечению (SRS) для системы обеспечения контура безопасности. В нем описываются область действия системы, как функциональные, так и нефункциональные требования к программному обеспечению, конструктивные ограничения и системные интерфейсы.
          Продукт должен обеспечить корректное функционирование для любого коммерческого предприятия.

        \item \begin{large} \textbf{Границы применения} \end{large} \newline
        Система обеспечения контура безопасности -система, представляющая собой комплекс программных и технических средств, необходимых для поддержания санкционированного доступа в помещения в охраняемых зонах.
        СОКБ должна предоставить доступ лицу, имеющему соответствующий пропуск. Лица, не имеющие пропуска, не должны иметь возможности доступа к объекту.  
        Программное обеспечение должно функционировать на двух уровнях безопасности. Первый уровень безопасности представляет собой установленный при входе в помещение турникет, считывающий с пропуска информацию о прошедшем лице. Вторым уровнем безопасности являются доступ к рабочему отделу организации. Также как и на первом уровне считывается информация о лице. При наличии соответствующих прав должен предоставиться доступ к отделу. Информация о доступах на обоих уровнях безопасности должна заноситься в базы данных. 
        Использование данной системы должно значительно снизить риск несанкционированного доступа, способного нанести серьезный материальный ущерб. 
        
        \item \begin{large} \textbf{Определения, сокращения, термины} \end{large} \newline
          \begin{tabular}{ | l | l |}
            \hline
            Аббривиатура & Расшифровка \\ \hline
            СОКБ & Система обеспечения контура безопасности \\
            АРМ & Автоматизированное рабочее место \\
            ЗПБ & Зона повышенной безопасности \\
            УД & Уровень доступа \\
            КСБ & Комплекс системы безопасности \\
            СУРВ & Система учета рабочего времени \\
            СКУД & Система контроля и управления доступом \\
            \hline
          \end{tabular}
        \item Ссылки
        \item Краткий обзор
    \end{enumerate} 
  \item  Общее описание
  	\begin{enumerate}
  		\item Описание изделия
  			\begin{enumerate}
  				\item Интерфейсы системы
  				\item Интерфейсы пользователя
  				\item Интерфейсы аппаратных средств ЭВМ
  				\item Интерфейсы программного обеспечения
  				\item Интерфейсы коммуникаций
  				\item Ограничения памяти
  				\item Действия
  				\item Требования настройки рабочих мест
  			\end{enumerate}
  		\item Функции изделия
  		\item Характеристики пользователей
  		\item Ограничения
  		\item Предложения и зависимости
  		\item Поднаборы требований(распределение требований)
  	\end{enumerate}
  \item Детальные требования
  	\begin{enumerate}
  		\item \begin{large} \textbf{Внешние интерфейсы} \end{large} \newline
  		  На первом уровне имеется 2 картоприемника, каждый из которых соединен посредством канала связи с турникетом.
  		  На вход картоприемник должен получить карту, с которой считывает необходимую информацию, а также заносит новые данные в базу данных. На выходе формируется ответ о наличии прав доступа и отсылает его турникету. Турникет должен быть связан с двумя картоприемниками. Картоприемник №1,расположенный на входе, при наличии прав доступа должен разрешить вход 1 человека. Картоприемник №2, расположенный на выходе, при наличии прав должен разрешить выход 1 человека.
  		     В обычном ситуации турникет должен находиться в одном из 3х состояний: \newline
    		  $\circmark $  вход и выход Запрещен. \newline
    		  $\circmark $ Вход разрешен, выход запрещен\newline
  	    	  $\circmark $ Вход запрещен, выход запрещен\newline
  		     В аварийном ситуации турникет должен находиться в состоянии:
  		      $\circmark $  вход и выход Разрешен. \newline
  		  Турникет имеет 3 индикатора. 
  		  Индикатор №1 имеет форму стрелочки влево и сообщает о том, что разрещен выход.
  		  Индикатор №2 имеет форму креста и сообщает о том, что запрещены вход и выход
  		  Индикатор №3 имеет форму стрелочки вправо и сообщает о том, что разрешен вход
  		  В Аварийной ситуации активны индикаторы №1 и №3.
  		  На втором уровне на входе в отдел имеется 1 картоприемник, который соединен каналом связи с дверью. При считывании данных с карты информация заносится в базу данных. При наличии прав доступа дверь открывается. При выходе из отдела имеется выключатель, открывающий дверь.  
  		\item Функции
  		   
  		\item Требования исполнения
  		\item Требования логики базы данных
  		\item Ограничения проекта
  		\item Характеристики программного обеспечения системы
  			\begin{enumerate}
  				\item Надежность
  				\item Эксплуатационная готовность
  				\item Безопасность
  				\item Ремонтопригодность
  				\item Переносимость
  			\end{enumerate}
  		\item Структурирование детальных требований
  			\begin{enumerate}
  				\item Режим системы
  				\item Классы пользователей
  				\item Объекты
  				\item Особенности
  				\item Воздействие
  				\item Реакция
  				\item Функциональные иерархии
  				\item Дополнительные комментарии
  			\end{enumerate}
  	\end{enumerate}
\end{enumerate}
 \end{document}